
% 
% $Description: Author guidelines and sample document in LaTeX 2.09$
% 
% $Author: Amaury Hernandez Aguila $
% $Date: 2013/6/4 15:20:59 $
% $Revision: 1.4 $
% 

\documentclass[times, 10pt,twocolumn, a4paper]{article}
\usepackage{latex8}
\usepackage{times}
\usepackage{amsmath}
\usepackage{graphicx}
\usepackage[spanish]{babel}

% \documentstyle[times,art10,twocolumn,latex8]{article}

% -------------------------------------------------------------------------
% take the % away on next line to produce the final camera-ready version
\pagestyle{empty}

% -------------------------------------------------------------------------
\begin{document}

\title{Sistema de Recomendaci\'on de Librer\'ias de C\'odigo Fuente
  basado en Tf-idf y Similitud por Coseno}

\author{Amaury Hern\'andez \'Aguila\\
  Instituto Tecnol\'ogico de Tijuana\\
  Maestr\'ia en Ciencias Computacionales\\
  amherag@gmail.com\\
  % For a paper whose authors are all at the same institution,
  % omit the following lines up until the closing ``}''.
  % Additional authors and addresses can be added with ``\and'',
  % just like the second author.
}

\maketitle
\thispagestyle{empty}

\begin{abstract}

%\begin{figure}[ht!]
%\centering
%\includegraphics[width=40mm]{logo.png}
%\caption{MetaLambda Logo}
%\label{metalambda}
%\end{figure}

%  \({n+1\choose 3}\)
%  \(\theta\)
% \ref{metalambda}

%\(a  \cdot b = \parallel a \parallel. \parallel a \parallel cos \theta\)



  Se cre\'o un sistema de recomendaci\'on de librer\'ias de c\'odigo
  fuente de Common Lisp. El sistema tambi\'en est\'a hecho en Common
  Lisp. Se utilizan los m\'etodos tf-idf (\textit{Term frequency -
    Inverse document frequency}) y similitud de cosenos para
  comparar los c\'odigos fuentes y las documentaciones entre paquetes
  (librer\'ias). Adem\'as se utiliza una adaptaci\'on del tf-idf y
  similitud de cosenos que calcula la similitud entre dos paquetes
  tomando en consideraci\'on el uso de otras librer\'ias dentro de
  este paquete. El resultado del sistema de recomendaci\'on es una
  lista de paquetes similares al paquete del usuario junto con las librer\'as recomendadas.
\end{abstract}

\begin{center}
\textbf{Palabras clave:} sistema de recomendaci\'on, similitud de
c\'odigo fuente, tf-idf, similitud por coseno, metalambda
\end{center}

% -------------------------------------------------------------------------
\Section{Introducci\'on}

La mayor\'ia de los sistemas cuyo enfoque es encontrar la similitud
entre c\'odigos fuente tienen como finalidad la detecci\'on de
plagiarismo. En este trabajo se presenta una forma de utilizar estas
medidas de similitud para hacer un sistema de recomendaci\'on de
c\'odigo fuente, espec\'ificamente de librer\'ias de Common Lisp.

Un desarrollador puede utilizar este sistema de recomendaci\'on para
encontrar paquetes que sean similares al suyo, o paquetes que
podr\'ian ser de utilidad para su proyecto.

La utilidad de un sistema de recomendaci\'on como este s\'olamente se
presenta si se tiene a nuestra disposici\'on una gran colecci\'on de
librer\'as las cuales podamos procesar f\'acilmente. Especialmente,
nuestro sistema de recomendaci\'on necesita una forma de determinar de
qu\'e librer\'ia proviene cualquier definici\'on (funciones,
variables, etc.) siendo usada en un paquete. Esto no presenta un
problema para nuestro sistema de recomendaci\'on, ya que Common Lisp
nos da la posibilidad de obtener el paquete de donde provienen los
s\'imbolos en un programa.

El sistema de recomendaci\'on utiliza el promedio de tres m\'etricas para
determinar la similitud entre un paquete y otro. De cada paquete se
extrae las documentaciones de sus funciones, su c\'odigo fuente
excluyendo las documentaciones de sus funciones, y la usabilidad de
cada librer\'ia en el paquete. \'Este \'ultimo se refiere a la
cantidad de veces que un paquete est\'a usando alguna funci\'on,
m\'etodo, clase, variable, etc. (o s\'imbolo, en t\'erminos de Common
Lisp) de cierta librer\'ia; entre mayor sea la cantidad de s\'imbolos
externos siendo usados en el paquete, mayor ser\'a la usabilidad de
esa librer\'ia dentro de ese paquete.

La similitud entre documentaciones y c\'odigo fuente de los paquetes
se obtiene utilizando una implementaci\'on de los m\'etodos tf-idf y
similitud de cosenos. Utilizando tf-idf se obtiene un vector con las
frecuencias de las palabras en cada documentaci\'on y c\'odigo fuente, y con
similitud de cosenos obtenemos la similitud entre los vectores
obtenidos en el paso anterior con tf-idf.

El utilizar estas tres m\'etricas diferentes ayuda enormemente a que
el sistema de recomendaci\'on reconozca mejor las necesidades del
paquete. Es frecuente que los desarrolladores olviden u omitan
deliberativamente la documentaci\'on de sus proyectos, sin embargo,
qu\'e paquetes est\'an siendo utilizados son datos que est\'an
\'implicitos en el c\'odigo, as\'i como el c\'odigo mismo.

\Section{Motivacion}
\label{Motivacion}

Este proyecto sirve como un prototipo de un sistema de recomendaci\'on
que ser\'a implementado en la plataforma de programaci\'on en l\'inea de
MetaLambda.

MetaLambda es esencialmente un servicio de \textit{hosting} y una
plataforma de desarrollo de software de prop\'osito general
colaborativa.

Las librer\'ias en MetaLambda son construidas por los mismos usuarios
de la plataforma, por lo que la colecci\'on de paquetes que pueden ser
incluidas en los proyectos de los usuarios es enorme. Varias
herramientas est\'an siendo desarrolladas para facilitarle la
b\'usqueda de librer\'ias y paquetes al usuario, como un motor de
b\'usqueda, y este mismo sistema de recomendaci\'ion de librer\'ias.

Common Lisp es el mismo lenguaje de programaci\'on que est\'a siendo
usado para desarrollar MetaLambda, as\'i que se decidi\'o utilizarlo
tambi\'en para nuestro sistema de
recomendaci\'on.

% -------------------------------------------------------------------------
\Section{Trabajos Previos}

Hay numerosos trabajos relacionados con la detecci\'on de similitud en
c\'odigos fuente, sin embargo, hay dos trabajos que tienen bastante
similitud con mi trabajo, CLAN \cite{ml1} y MUDABlue \cite{ml2} cuyo objetivo es encontrar software similar
dentro de un repositorio. La finalidad de nuestro sistema es diferente a la
de estos dos trabajos, pero comparten similitud en los procesos, ya
que todos se encargan de encontrar software similar.

\Section{Modelo Propuesto}

El modelo que proponemos en este trabajo se basa en en los m\'etodos
de tf-idf y similitud de cosenos para encontrar la similitud entre
paquetes de software. Las caracter\'isticas utilizadas que ser\'ian m\'as
intuitivas son la comparaci\'on entre las documentaciones de los
paquetes y la comparaci\'on entre los c\'odigos fuentes. La tercera
m\'etrica es la menos intuitiva y m\'as innovadora, ya que se hace una
adaptaci\'on de los m\'etodos tf-idf y similitud de cosenos tomando
como datos la frecuencia de utilizaci\'on de funciones de las
diferentes librer\'ias incluidas en el paquete.

Otras caracter\'isticas pueden ser usadas en este sistema, pero se
decidi\'o tomar estas tres para iniciar este prototipo, ya que
consideramos que son las m\'as relevantes.

% -------------------------------------------------------------------------
\SubSection{Tf-idf}

El m\'etodo tf-idf, descrito en \cite{ml3}, tiene como finalidad determinar la relevancia de un
t\'ermino en un documento con respecto a una colecci\'on de documentos
o \textit{corpus}. Este m\'etodo f\'acilmente puede ser adaptado para
que calcule un vector con la relevancia de cada una de las palabras
presentes en cierto documento, y de esta forma se pueden realizar
comparaciones entre dos documentos.

Este m\'etodo consiste en en el producto de dos estad\'isticas \cite{ml5}: \textit{tf} o
\textit{term frequency} e \textit{idf} o \textit{inverse document
  frequency}. Para la estad\'istica \textit{tf}, el m\'etodo debe
calcular la frecuencia de un t\'ermino en un s\'olo documento. La
forma m\'as simple ser\'ia determinar las veces que un t\'ermino
aparece en un documento. Sin embargo, existen otras posibilidades
\cite{ml6}, como una forma booleana, en la si el t\'ermino est\'a
presente en el documento, sin importar el n\'umero de veces, \(tf(t,d)
= 1\), donde \(t\) es el t\'ermino y \(d\) es el documento. Este es el
camino que tomamos para determinar la frecuencia \(tf\) en nuestra
m\'etodo. La estad\'istica \textit{idf} es una medida que determina
qu\'e tan com\'un es cierto t\'ermino en todo el \textit{corpus}. Para
calcular esta medida utilizamos la ecuaci\'on

\begin{align}
  idf(t,D) = log {\vert D \vert \over \vert \{ d\in D : t \in d\}\vert }
\end{align}

en donde \(t\) es el t\'ermino del cual deseamos obtener la relevancia
dentro del \textit{corpus}, \(D\), \(\vert D \) es la cardinalidad de
\(D\) o el n\'umero total de documentos, y \( \vert d \in D : t \in d
\vert \) es el n\'umero de documentos dentro de \(D\) en los que
ocurre el t\'ermino \(t\). \( tf(t, d) \ne 0 \) ya que ocurrir\'ia una
divisi\'on entre 0; una soluci\'on com\'un a esto es ajustar la
f\'ormula a \( \vert d \in D : t \in d \vert + 1 \), pero en nuestro
m\'etodo decidimos implementar la condici\'on de que si \( tf(t, d) =
0 \), entonces \( tf := 1 \).

La similitud entre dos c\'odigos fuentes y las documentaciones de dos
paquetes es llevada a cabo utilizando una implementaci\'on del
m\'etodo tf-idf.

Para medir la relevancia entre el uso de diferentes
librer\'ias en dos paquetes diferentes, se usa este mismo m\'etodo,
pero adapt\'andolo para que en lugar de utilizar t\'erminos en formato
de cadena de caracteres, utilice el idetificador o nombre de un
paquete. La frecuencia del uso de estas librer\'as es obtenida
determinando el n\'umero de funciones de cada librer\'ia presentes en
el c\'odigo fuente del paquete. As'\i, si una librer\'ia
que est\'a siendo incluida en el paquete, pero ninguna de sus
funciones forma parte de la estructura del c\'odigo fuente,
ser\'ia equivalente a una situaci\'on en la que no est\'a siendo usada
dicha librer\'ia. Al igual que la implementaci\'on del m\'etodo tf-idf
para medir la similitud entre las documentaciones y el c\'odigo
fuente, esta implementaci\'on regresa un vector que califica el uso de
cada librer\'ia presente en un paquete.

\SubSection{Similitud por coseno}

Habiendo obtenido los vectores resultantes de la aplicaci\'on del
m\'etodo tf-idf, podemos utilizar el m\'etodo de similitud por coseno
para determinar qu\'e tan similares son dos paquetes \cite{ml4} de
acuerdo a las tres caracter\'isticas que estamos tomando en
consideraci\'on.

La similitud por coseno puede ser obtenida a trav\'es de la ecuaci\'on

\begin{equation} \label{similarity}
  cos(\theta ) = {{A \cdot B} \over {\Vert A \Vert \Vert B \Vert}}
\end{equation}

donde \( cos(\theta ) \) representa a la similitud entre los vectores
\(A\) y \(B\), \(A \cdot B\) es el producto punto de ambos vectores y
\({\Vert A \Vert \Vert B \Vert}\) es el producto de la magnitud de los
vectores \(A\) y \(B\). El producto punto de dos vectores \(a\) y
\(b\) esta definido como

\begin{align}
  {a \cdot b} = \sum_{i=1}^n a_i b_i = a_1 b_1 + a_2 b_2 + \dots + a_n
  b_n = a b^T
\end{align}

y la magnitud de un vector \(x\) como

\begin{align}
  {\Vert x \Vert} = \sqrt{x_1 ^2 + x_2 ^2 + \dots + x_n ^2}.
\end{align}

De esta forma, podemos reescribir \eqref{similarity} como

\begin{equation}
  cos(\theta ) = { {\sum_{i=1}^n A_i B_i} \over {\sqrt{\sum_{i=1}^n
        (A_i) ^ 2} \sqrt{\sum_{i=1}^n (B_i) ^ 2}}}
\end{equation}

y resulta f\'acil implementar el m\'etodo utilizando ciclos que iteren
por cada uno de nuestros vectores resultantes del \textit{tf-idf}
aplicado a los documentos siendo comparados.

Una vez calculada la similitud por coseno entre los vectores de dos
documentos, el resultado es un escalar \(r \in \Re\), donde \( 0 \leq
r \leq 1 \), y \(r\) nos representa una calificaci\'on de la similitud
entre los dos documentos.

\SubSection{Promedio de calificaciones}
\label{Promedio}

En nuestro m\'etodo aplicamos similitud por coseno a los vectores
resultantes del \textit{tf-idf}, pero este m\'etodo tiene que ser
aplicado tres veces, una vez a cada una de nuestras
caracter\'isticas. Por lo tanto, el resultado sigue siendo un vector
que contiene las tres calificaciones por cada una de las
caracter\'isticas.

Para poder encontrar las librer\'ias que tengan mayor similitud con
nuestro paquete, es m\'as conveniente trabajar con una calificaci\'on
escalar que vectorial. En este trabajo se opt\'o por utilizar el
promedio de las tres calificaciones finales. As\'i que

\begin{equation}
  {similitud(P_D, P_S, P_L)} = {{cos(\theta_D) + cos(\theta_S) +
      cos(\theta_L)} \over 3},
\end{equation}

donde \(P_D\), \(P_S\) y \(P_L\) son la documentaci\'on, el c\'odigo
fuente y la frecuencia de uso de las librer\'ias en los paquetes
siendo comparados. As\'i, obtenemos como resultado final la
similitud \(s \in \Re\), donde \( 0 \leq s \leq 1 \).

Podr\'iamos utilizar otro m\'etodo para reducir nuestro vector de
calificaciones a un escalar, pero este m\'etodo resulta ser suficiente
para este prototipo. Notablemente, podr\'iamos realizar esta misma
operaci\'on de un promedio de las calificaciones, pero utilizando una
ponderaci\'on. Los pesos utilizados en la ponderaci\'on pueden ser
optimizados mediante alg\'un algoritmo de optimizaci\'on, tal como un
algoritmo gen\'etico, y esto llevar\'ia a un sistema de
recomendaci\'on m\'as efectivo.

\SubSection{Recomendaci\'on de paquetes}

Al darle al sistema de recomendaci\'on nuestro paquete, esperamos que
nos recomiende otros paquetes o librer\'ias. Utilizando el promedio de
calificaciones podemos obtener una lista de calificaciones de
similitud de nuestro paquete con respecto a cada uno de los paquetes
en el \textit{corpus}. Sin embargo, la recomendaci\'on de los paquetes
que son similares al nuestro puede no ser la mejor respuesta para el
usuario, ya que el sistema simplemente nos est\'a estableciendo
cu\'ales paquetes son los m\'as similares a nuestro paquete. La
soluci\'on que se toma es dar como resultado las librer\'ias que
utilizaron los paquetes similares al nuestro, excluyendo aquellos
paquetes que ya estamos utilizando en nuestro proyecto.

\Section{Trabajos futuros}
\label{Futuros}

El estado actual de este sistema de recomendaci\'on no es apto para
ser aplicado a un proyecto en producci\'on. Hay varios problemas que
son notorios.

Un gran problema es que los vectores generados por el m\'etodo
\textit{tf-idf} conservan las calificaciones de aquellos t\'erminos
que no estuvieron presentes en el documento, pero que s\'i lo
estuvieron en el \textit{corpus}. Esto podr\'ia ser beneficioso cuando
se aplica a la caracter\'istica de la frecuencia de uso de librer\'ias
en el paquete, ya que el que dos paquetes no usen cierta librer\'ia
puede ser considerada una semejanza. Sin embargo, no deber\'ian ser de
utilidad alguna cuando el m\'etodo es aplicado a las documentaciones y
a los c\'odigos fuente de los paquetes.

Otro problema es que el sistema de recomendaci\'on presenta en los
resultados cada uno de los paquetes dentro del \textit{corpus}
excluyendo al paquete siendo comparado. En una colecci\'on de paquetes
peque\~na no representar\'ia un problema grande pero, conforme crece
la colecci\'on de paquetes, la lista de resultados podr\'ia ser
dif\'icil de interpretar por el usuario.

En la lista de paquetes que se arroja como resultado del sistema, se
le presentan al usuario cada uno de los paquetes dentro de la
colecci\'on de paquetes junto con su porcentaje de similaridad. Como
se ha mencionado dentro del planteamiento del sistema, estos paquetes
similares no representan un resultado de utilidad para el usuario. Una
mejor estimaci\'on de recomendaciones para nuestro paquete puede ser
la suma de los productos de la semejanza de cada uno de los paquetes
del \textit{corpus} con respecto a nuestro
paquete, con el \(tf-idf(t,d,D)\) de cada una de las librer\'ias
siendo usadas por cada uno de los paquetes similares, donde \(t\) es cada uno de las
librer\'ias que est\'an en uso en el paquete similar al nuestro, \(d\)
es cada uno de los paquetes similares al nuestro, y \(D\) es nuestra
colecci\'on de paquetes o \textit{corpus}.

Una vez que el sistema de recomendaci\'on est\'e en una forma m\'as
robusta, debe ser implementado en la plataforma de MetaLambda,
explicada en la secci\'on \ref{Motivacion}. Como caracter\'isticas
m\'as importantes, nuestro sistema de recomendaci\'on debe ser
adaptado para trabajar con la base de datos de grafos \textit{Titan}
\cite{ml7}, y el sistema deber\'ia extraer autom\'aticamente los
paquetes de los cuales pertenecen las funciones que forman parte de la
estructura de los paquetes - tarea que es f\'acilmente lograda gracias
a Common Lisp.

Sin duda alguna, el trabajo m\'as urgente es crear un m\'etodo que nos
permita medir la efectividad del sistema de recomendaci\'on. Un
m\'etodo propuesto por el Doctor Mario Garc\'ia Valdez, del Instituto
Tecnol\'ogico de Tijuana, es tomar una muestra de paquetes del
\textit{corpus} y eliminar el uso de algunas librer\'ias en los
paquetes de la muestra. Si el sistema es capaz de recomendar las
librer\'ias que fueron eliminadas, entonces se encuentra en la
direcci\'on correcta. Podemos utilizar el porcentaje de
recomendaci\'on descrito en esta misma secci\'on como medida de rendimiento.

\Section{Conclusiones}

La pieza central de nuestro sistema de recomendaci\'on es el m\'etodo
\textit{tf-idf}. Gracias a este m\'etodo resulta muy f\'acil encontrar
similitudes entre documentos tomando en consideraci\'on una gran
variedad de caracter\'isticas, y puede ser adaptado de una forma
muy simple. Adem\'as de esta simplicidad, el \textit{tf-idf} es muy
versatil, ya que hay muchas posibilidades de encontrar las
estad\'isticas \textit{tf} y \textit{idf}. Estamos seguros de que la
modificaci\'on de este m\'etodo es vital para el descubrimiento de
maneras m\'as efectivas de obtener mejores resultados.

Adem\'as de la modificaci\'on del m\'etodo \textit{tf-idf}, un \'area
de investigaci\'on que se abre para nosotros es la elecci\'on de las
caracter\'isticas a considerar. Gracias a las grandes cantidades de
informaci\'on sobre el c\'odigo fuente y los usuarios que es
almacenada en la plataforma MetaLambda, se abre la posibilidad de
crear un sistema de recomendaci\'on m\'as robusto y refinado.

Una parte importante del sistema de recomendaci\'on a la cual no se le
dio la importancia necesaria fue a la secci\'on \ref{Promedio}. El
m\'etodo elegido - simplemente promediar los valores de las
calificaciones del \textit{tf-idf} en el vector - es una soluci\'on
demasiado burda. Podr\'iamos utilizar alg\'un m\'etodo de
optimizaci\'on para encontrar pesos para una ponderaci\'on de
las calificaciones que nos den resultados m\'as significativos. Para
poder lograr esto, es indispensable establecer un m\'etodo que nos
ayude a probar el sistema, un trabajo que se encuentra actualmente en
desarrollo, tal como mencionamos en la secci\'on \ref{Futuros}.

Como conclusi\'on general, hemos podido desarrollar un prototipo que
cumple satisfactoriamente con las necesidades iniciales de un sistema
de recomendaci\'on de librer\'ias de c\'odigo fuente para la
plataforma de MetaLambda.

\Section{Agradecimientos}

Quiero agradecer el apoyo de mi novia Elizabeth Quetzali Madera
Hern\'andez, quien me ayud\'o con muchas ideas para mejorar este
sistema de recomendaci\'on, adem\'as de ser un miembro vital del
equipo de MetaLambda.

Tambi\'en ha sido de gran importancia para m\'i tener el apoyo de
CONACYT, el cual ha sido definitivo para que haya podido continuar con
mis estudios de posgrado, as\'i como el apoyo de todos los catedr\'aticos del
Instituto Tecnol\'ogico de Tijuana, especialmente al Doctor Mario
Garc\'ia Valdez, quien contribuy\'o enormemente con su direcci\'on en
este proyecto.

% -------------------------------------------------------------------------
%\footnote
%{%
%  Or, better still, try to avoid footnotes altogether.  To help your
%  readers, avoid using footnotes altogether and include necessary
%  peripheral observations in the text (within parentheses, if you
%  prefer, as in this sentence).
%}

% -------------------------------------------------------------------------
%\nocite{ml1,ml2}
\bibliographystyle{latex8}
\bibliography{latex8}

\end{document}
