\section{Introduction}

There are many tools that can be used to forecast the next values of a time-series, and many others specialized in forecasting the more complex financial time-series. Since many decades ago, financial traders have implemented statistical models to forecast the time-series, and a field called Technical Analysis was created. This type of market analysis is characterized by using past financial data to describe certain aspects of a market, such as its volatility, trend, and momentum. These tools receive the name of Technical Indicators (TI). Some examples of TI are those created by Welles Wilder, such as Relative Strength Index (RSI) and Average Directional Index \cite{wilder1978new}, and an example of an oscillator (a type of TI which usually tells if a market is overbought or oversold) is the Stochastic Oscillator \cite{schirding1984stochastic}, created by George Lane. In contrast to Technical Analysis, Fundamental Analysis relies on the examination of the underlying forces that affect a particular financial market (or several of them), instead of just relying on the price movements in a time-series. A technique that can be used in Fundamental Analysis is to perform a group selection of companies, and analyze their economic behaviors to establish a general forecast for each of them. After performing this analysis, one can narrow the group of companies and decide on a proper trading strategy to follow according to the arrived conclusions. To analyze a company, one can resort to examine its business plan and its management.

A more modern approach to perform these analyses is to use Machine Learning (ML) techniques. With ML, one can use inputs provided by Technical or Fundamental Indicators and create regression models. Examples of regression techniques are autoregression \cite{burg1968new}, symbolic regression \cite{billard2002symbolic}, and linear regression \cite{kutner2004applied}. Other more elaborated techniques exist in ML for regression or curve-fitting tasks, such as the use of Artificial Neural Networks \cite{melin2007hybrid} and Support Vector Regression \cite{basak2007support}. But a problem that often arises with these models is that they can't take into account emergent phenomena: these models learn from past behaviors and can't predict what they haven't "seen." An approach that has shown excellent results in simulating these behaviors is Agent-based Modeling (ABM). Citing the work by Bonabeau \cite{bonabeau2002agent}, ABM is, "by its very nature, the canonical approach to modeling emergent phenomena," and for this reason, it is a suitable tool to model complex systems \cite{jennings2001agent}, such as financial markets. A concise definition of ABM can be found in the work by Gilbert \cite{gilbert2008agent}: "Agent-based modeling is a form of computational social science... One creates some kind of simplified representation of 'social reality' that serves to express as clearly as possible the way in which one believes that reality operates." In particular, ABM has been proved to be useful to analyze price stability in financial markets \cite{Pellizzari2007}, scientometrics and domain visualization \cite{Niazi2011}, and in social sciences in general \cite{gilbert2008agent}, among others.

This work proposes a method based on ABM to construct a Decision Support System (DSS) that aids traders on the process of trading financial markets. DSS is not a well defined term, as for some researchers is just an interactive system for managers, and for others the focus is on understanding and improving the decision process \cite{keen1980decision}. Since its inception, DSS have been created to aid managers in business-related areas \cite{Sprague1980} \cite{little1979decision}, and one can see specialized cases where DSS are used to support the decision process when forecasting financial markets, such as in the work by Tsang, Yung and Li \cite{Tsang2004}. The proposed method is aimed at improving the decision process of trading financial markets: the trader receives a recommendation of what financial market to trade, as in the work by Brown, Pelosi, and Dirska \cite{brown2013dynamic}. But the created algorithm goes deeper than just giving recommendations; in the end, a novel method for performing regression was created, and the authors believe that it can be extended to perform classification tasks. The method uses a combination of Fuzzy Inference Systems (FIS), Genetic Programming (GP) \cite{poli2008field} \cite{Koza1992} and Multi-agent Systems (MAS) \cite{Shoham2009} to produce models that can explain the behavior of complex systems such as that of a financial market.

The differences between ABM and MAS are subtle, but have to be noted before proceeding. According to Niazi, and Hussain \cite{Niazi_2011}, MAS is a sub-domain of ABM; the principal objective of ABM is to provide an explanation of a phenomenon through the interaction of agents, while MAS provides a specific application of ABM in order to solve a practical problem. The MAS developed in this work belongs to a class of Coalitional Games called Constant-sum game \cite{Shoham2009}, and, in particular, it is a zero-sum game. Agents in the proposed method are selfish agents that don't intervene with other agents, but there's a higher order process that is dictating how they must organize. In this case, the collective work of every agent must give a "zero-sum": the sum of their forces must be equal to the observed price, or in other words, they must adjust to the real prices. To achieve this, each agent uses a FIS to act intelligently, and its Membershif Functions (MF) are found using GP. Specifically, the predicates and consequents of the FIS are generated using sum of sines that are found using a modified algorithm based on GP. The modified algorithm implements communities of agents that perform different operations among their agents: crossover, migration, replace and mutation. In the end, the developed algorithm gives as result the community of agents that obtained the absolute value of the lowest Mean-Squared Error (MSE) plus the sum of its agents scores (explained in Section \ref{proposed-method}). At first, the authors of this work had the hypothesis that the developed system would enable the user to examine the MF of each agents' FIS to develop new economic theories, but, even though the system achieved remarkable results, the generated MF are too chaotic to be interpreted with ease.

The structure of this work is as follows: in Section \ref{related-work} an extensive background of related works is given. Special attention was given to this section in order to provide the necessary foundations for the Proposed Method, as it is a new algorithm based on many different technologies and ideas. In Section \ref{proposed-method} the theory of the Proposed Method is given, explaining every aspect of its specification. A number of extensive experiments are presented in Section \ref{experiments-and-results}, along with their results. The experiments were carefully designed in order to provide enough information to support the efficacy of the Proposed Method. Conclusions about the presented work are given in Section \ref{conclusions}, and a brief description about the work that is planned to be done in a future is given in Section \ref{future-work}.
  