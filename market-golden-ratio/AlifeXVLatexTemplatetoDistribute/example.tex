\documentclass[letterpaper]{article}
\usepackage{natbib,alifeconf}
\usepackage{hyperref}

\title{Instructions for Authors: ALIFEXV Proceedings}
\author{First Author$^{1}$, Second Author$^{1,2}$ \and Third Author$^2$ \\
\mbox{}\\
$^1$First affiliation  \\
$^2$Second affiliation \\
corresponding@author.email}


\begin{document}
\maketitle

\begin{abstract}
This paper describes the formatting requirements for papers submitted to the 15th International Conference on the Synthesis and Simulation of Living Systems (ALIFEXV). MIT Press will publish the proceedings in a single online open-access volume.  The proceedings will be assembled directly from Portable Document Format (PDF) files furnished by the authors. To ensure that all articles in the on-line publication have a uniform appearance, authors must adhere to the following instructions. The publication style used here is identical to the style of the ALIFE 14 conference to ensure uniformity over both conferences.
\end{abstract}

\section{Formatting Instructions}

Basically, the rule is � make your paper look like the example paper which is 
produced using the LaTeX style file. So, if you're a LaTeX user that's easy � 
use the LaTeX style file provided (this). If you're a Word user use the  
Word template. 
The page limit is 8 pages for a full paper, or 2 pages for an abstract. 
Your submission must be converted to Portable Document Format (PDF). 
Please be sure to use highest portability and quality options. Papers that 
significantly deviate from these instructions will not be included.


\subsection{Page Format}

\subsubsection{Paper, margins and columns.} 	
Your paper must be formatted in two-column 
format for Letter paper (8.5 x 11 inch). The total printed area on all pages is 
a maximum of 7 inches wide and 9 inches tall. For Letter paper the margins 
will therefore be as follows:
\begin{itemize}
{\item Top margin: 0.75 inch.}
{\item Left margin: 0.75 inch.}
{\item Right margin: 0.75 inch.}
{\item Bottom margin: 1.25 inch.}
\end{itemize}
Papers that deviate from these measurements will not be published. 
(These measurements apply only to Letter paper. Papers formatted for A4 
paper are unacceptable.
To ensure maximum readability, your paper must include two columns 
with a 0.38 inch gutter of white space between the two columns. This 
will mean that the printed width of each column will be 3.31 inches. 
Start all pages (except the first) directly under the top margin. 
(See the next section for instructions on formatting the title page.) 

\subsubsection{Typeface and font sizes.}
For the purpose of uniformity, 
format your paper in a Type 1 Times Roman PostScript. 
The sizes of fonts in the LaTeX style file are given in table 1 below. 
Using Times New Roman in Word with these font sizes looks a little larger 
in comparison: So to keep the appearance similar to the LaTeX example, 
the Word template uses the font sizes listed. \\

\begin{table} [h]
\begin{tabular}{| l | l | l |}
  \hline			
  & \multicolumn{2}{|c|}{Font size (and line spacing)} \\  
  \hline			
						& LaTeX style   	&  Word template \\
						& file (Computer 	& (Times New \\
						& Modern 		  	& Roman)\\   
						& Roman)  			& 		\\   
  \hline
   Main (normal) text 	& 10 (11) 			& 9.5 (10.5) \\
  \hline
   Title				& 14 (16)			& 14 (16) \\
  \hline
  Section heading		& 12 (14)			& 12 (14) \\
  \hline
  Subheading			& 11 (12)			& 10.5 (11.5) \\
  \hline  
  Subsubheading			& 10 (11)			& 9.5 (10.5) \\
  \hline  
  ``Abstract'' heading	& 10 (11)			& 10 (11) \\
  \hline  
  Abstract				& 9 (10)			& 8.5 (9.5) \\
  \hline  
  References			& 9 (10)			& 8.5 (9.5) \\
  \hline  
  Text in table/figure	& $\geq$ 9			& $\geq$ 8.5 \\
  \hline    
\end{tabular}
\caption{Font sizes for LaTeX and Word typefaces}
\end{table}

\subsubsection{Page numbers.} 
Do not include page numbers on your paper. Actual page 
numbers will be assigned by the publisher.


\subsection{Title}
The title appears near the top of the first page, 
centered over both columns with 42-point leading above. 
Author�s names should appear below the title of the paper 
(with 12 point leading above and below), along with affiliation(s) 
and complete address(es) (including electronic mail address if available) 
smaller font (see above). You should begin the two-column 
format when you come to the abstract. Any credits to a sponsoring 
agency should appear in the acknowledgments section, unless the agency 
requires different placement.

\subsection{Abstract}
The abstract must be placed at the beginning of the first column, 
indented ten points from the left and right margins. 
The title ``Abstract'' should appear in bold type, centered above the 
body of the abstract. The abstract itself is a one-paragraph summary 
{\bf describing the general thesis, its contributions and conclusion of 
your paper}. A reader should be able to learn the purpose of the paper 
and the reason for its importance from the abstract. Bear in mind the 
abstract will be used to advertise your talk in the conference program. 

\subsection{Text}
The main body of the paper follows the abstract. Font sizes above. 
Paragraph alignment should be fully justified.
Indent ten points when beginning a new paragraph, unless the 
paragraph begins directly below a heading or subheading. \\

\subsubsection{Citations.} 
Citations within the text should include the author's 
last name and year, for example \citep{CA2}.
Append lower-case letters to the year in cases of ambiguity. 
Multiple authors should be treated as follows: 
\citep{GE93}
or in the case of three or more authors list only the first author, 
followed by et al. (i.e., \citealt{Aguilar:2014}). 
Alternatively, citations by number are also acceptable if you are consistent. 
The following are examples of how to cite: a book \citep{Engelmore:1986}, a journal article \citep{Robinson:1980}, 
a magazine article \citep{Hasling:1983}, a
proceeding paper \citep{Chu:1993}, 
a university technical report \citep{Rice:1986}, 
a dissertation thesis \citep{Clancey:1979}, and an in press
publication (\citeauthor[in press]{Clancey:inpress}).


\subsubsection{Extracts.} Long quotations and extracts should be indented ten points from the left and right margins.
\begin{quotation}
This is an example of an extract or quotation. 
Note the indent on both sides. 
Quotation marks are not necessary if you offset the text in a block like this, 
and properly identify and cite the quotation in the text. 
\end{quotation}

\subsubsection{Footnotes.} Avoid footnotes as much as possible; they interrupt the 
reading of the text. When essential, they should be consecutively numbered 
throughout with superscript Arabic numbers. Footnotes should appear at the 
bottom of the page, separated from the text by a blank line space and a thin, 
half-point rule \footnote{This is an example of a footnote. Use sparingly!}. 

\subsection{Figures}
Figures, drawings, tables, and photographs should be placed throughout the 
paper near the place where they are first discussed. Do not group them 
together at the end of the paper. If placed at the top or bottom of the page, 
illustrations may run across both columns. Figures must not invade the top, 
bottom, or side margin areas. Insert figures using your page-formatting software. 
Number figures sequentially, for example, figure 1, and so on.
The illustration number and caption should appear under the illustration. 
Leave some space between the figure and the caption and surrounding type; 
1/4 inch should suffice. Ensure that each {\bf figure caption provides a concise 
explanation of what is shown in the figure}, including (if necessary) the model 
parameters that produced the results shown in the figure.  
{\bf Ensure also that the text in each figure is readable} (check the labels on the 
axes).

\section{Headings and Sections}

When necessary, headings should be used to separate major sections of your paper. 
An overabundance of headings will tend to make your paper look more like an 
outline than a paper.
First-level heads should be bold type, mixed case (initial capitals followed 
by lower case on all words except articles, conjunctions, and prepositions, 
which should appear entirely in lower case), with three-point leading, 
centered, with one blank line preceding them and three additional points 
of leading following them. 

\subsection{Sub-headings}
Second-level headings should be bold type, mixed case, with two-point 
leading, flush left, with one blank line preceding them and three 
additional points of leading following them. Do not skip a line between 
paragraphs.

\subsubsection{Third-level headings.}
Third-level headings should be run in with the text, bold type, mixed case, 
flush left, with six points of additional space preceding them and no 
additional points of leading following them.

\subsubsection{Acknowledgments.} 
The acknowledgments section, if included, appears after the main body 
of text and is headed ``Acknowledgments.'' This section includes 
acknowledgments of help from associates and colleagues, credits 
to sponsoring agencies, financial support, and permission to publish. 
Please try to limit acknowledgments to no more than three sentences.

\subsubsection{Appendices.} 
Any appendices follow the acknowledgments (if included or after the 
main body of text if no acknowledgments appear).

\subsubsection{References.} 
The references section should be labeled ``References'' and should 
appear at the end of the paper. A sample list of references is given 
at the end of these instructions. Please use a consistent format for 
references � see examples below and example paper. Poorly prepared or 
sloppy references reflect badly on the quality of your paper and your 
research. Please prepare complete and accurate citations. 

\section{Preparing Electronic Version of Your Paper}
Papers must be submitted in Portable Document Format (PDF), and delivered 
using the EasyChair submission framework accessible via the 
\href{url}{http://xva.life}
submission site (do not email papers). PDF files may be created using Adobe 
Systems Inc.'s PDFWriter or converted from postscript to PDF using Distiller 
or a similar product. Recent versions of Microsoft Office programs can also 
save a document in PDF format. If you do not have suitable PDF creation software,
many other free PDF converters can be found on-line (e.g. 
\href{url}{http://docmorph.nlm.nih.gov/docmorph/}). 
Unix/Linux conversion can be performed with the ps2pdf command.
Useful information for controlling the appearance of fonts in LaTeX may 
be found in Kendall Whitehouse's ``Creating Quality Adobe PDF Files from 
TeX with DVIPS'' (see 
e.g. \href{url}{http://frank.harvard.edu/~coldwell/dvips-pdf.html}).

\subsection{Style Files and Templates}
As a courtesy to authors, the ALIFEXV organizers created generic templates 
and style files that can be used to format two-column camera-ready copy
\footnote{This document was last revised on October 16 2015}. 
(Please read the formatting instructions!) You can retrieve these templates 
via \href{url}{http://xva.life}.
From this point, you should be able to follow the 
links to the specific information you require. 
The style files and templates have been tested only on a limited number 
of devices. They are therefore provided ``as is'' without any guarantee 
that they will work on your particular machine. If you are having trouble 
with the macros or templates, we suggest you contact an expert who is 
familiar with the particular hardware and software environment at your 
site for assistance. 

\section{Enquiries}
If you have any questions about the preparation or submission of your paper as instructed here, please email Program Chair (J. Mario Siqueiros, jmario.siqueiros@iimas.unam.mx).	

\bibliographystyle{apalike}
\bibliography{example}

\end{document}
